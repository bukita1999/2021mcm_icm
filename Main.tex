\documentclass{mcmthesis}
\mcmsetup{CTeX = false,   % 使用 CTeX 套装时,设置为 true
        tcn = 2001323, problem = A,
        sheet = true, titleinsheet = true, keywordsinsheet = true,
        titlepage = false, abstract = false}
\usepackage{palatino}
\usepackage{lipsum}
\usepackage{float}
\usepackage{geometry}
%===============设置正文和数学字体=============================
%有些字体需要安装一些字体文件,注意辨别。
%我参照 MCM论文集的字体 使用如下宏包来定制字体。

\usepackage{graphicx}

\usepackage{subfigure}
%设置段落之间的距离,若不需要删除或者注释掉即可。
\setlength\parskip{.5\baselineskip}
\newtheorem{definition}{Definition}[section]
%\def\abstractname{Summary}%可修改摘要名称
\usepackage{url}
\usepackage{indentfirst}
\setlength{\parindent}{2em}
\usepackage{siunitx}
\usepackage{chngpage}
\usepackage{array}
\usepackage{booktabs}
\usepackage{threeparttable}
\usepackage{longtable}
\usepackage[numbers,sort&compress]{natbib}
\usepackage{amsmath}
\numberwithin{figure}{section}
\numberwithin{table}{section}
%%% 实现参考文献标号在右上角
\newcommand{\upcite}[1]{\textsuperscript{\textsuperscript{\cite{#1}}}}
%然后引用的时候使用\upcite{}的格式(一般的正常引用格式为\cite{})



\usepackage{titletoc}
\titlecontents{section}[3cm]{\bf \large}{\contentslabel{2.8em}}{}{%
\titlerule*[0.5pc]{$\cdot$}\contentspage}%
\titlecontents{subsection}[4cm]{\normalsize}{\contentslabel{2.5em}}{}{%
\titlerule*[0.5pc]{$\cdot$}\contentspage}%
\titlecontents{subsubsection}[5.3cm]{\normalsize}{\contentslabel{3.0em}}{}{%
\titlerule*[0.5pc]{$\cdot$}\contentspage}%



\title{\large The Prediction to the Migration of Scottish Herring and Mackerel }
\author{}
\date{\today}
\geometry{left=2.5cm,right=2.5cm}

\begin{document}



\begin{abstract}
  In this paper, we have predicted the most possible locations of these two fish species in the next 50 years based on computer simulation methods. And we used a risk decision model to evaluate both strategies. 
  
  For question one: We selected the ocean range of \ang{52} N \textasciitilde \ang{62} N, \ang{0} W \textasciitilde \ang{10} W as the research object and divided the range into a $20 \times 10$ grid. We then calculate the average temperature within each grid. For some key grids, the method of polynomial fitting followed by interval limitation and noise processing is used. From the temperature data from 2000 to 2017, the average temperature change in the next 50 years is predicted. 
  Computer simulation was used to finally determine the most likely locations for the school of fish in the next 50 years based on water
temperature changes over the years.

  For question two: We found that the average water temperature rises by 0.075 degrees Celsius every 20 years, and the distribution of fish schools moves about one degree northward every 20 years. Then we counted the total fish content of the fishing grounds in Scotland every year. We think that when the value drops to a certain threshold, then small fisheries companies will not be able to harvest. The best situation will come into existence within 50 years and the worst situation will come up in 200 years. 
  
  For question three: Our goal is to maximize profits in the next 50 years. So, we have established empirical models of the cost and income of the two strategies, and then combined the distribution changes of the fish school to calculate the profit of the two strategies each year. We finally considered the advantages and disadvantages of both strategies. 
  
  For question four: After 50 years, a certain proportion of fish will enter the waters of the Danish Faroe Islands, but for small fisheries companies with limited operating radius, this impact is not significant.
  
\begin{keywords}
Migration Prediction; Computer Simulation Method; Polynomial Fitting; 
\\ \hspace*{1.2cm}Risk Decision
\end{keywords}
\end{abstract}
\maketitle
\pagestyle{empty}
\newpage                                                          %
%==================================================================
%====================生=成=目=录===================================
\begin{adjustwidth}{-1cm}{0cm}

\setcounter{tocdepth}{3}
\thispagestyle{empty}
\tableofcontents                                                  %

\end{adjustwidth}


\newpage

\pagestyle{fancy}

\setcounter{page}{1}
\section{Introduction}
\subsection{Background}

In recent years, as global warming has become more severe, seawater temperatures have also been significantly affected. Marine organisms are very sensitive to the temperature of seawater. Each organism has a critical temperature range for its growth and reproduction. The temperature controls the enzyme reactions that affect digestion through hormones and nerves. At the same time, the increase in temperature will increase the toxicity of toxic substances in the water, driving fish away from the originally suitable sea area to the sea area with a more suitable temperature.\cite{REF1}

Taking North Sea Cod as an example, the average distribution of cod in the North Sea has also changed significantly over the past three decades, most of which have shifted to more northern and deeper waters. A strong influence on this phenomenon is the apparent warming trend in the North Sea. At the same time, this phenomenon has also led to a plunge in catches by the border countries in their original fishing grounds. However, the northward shift of fish stocks in the world has damaged some fishing grounds, but also spawned other fishing grounds.
% 近年来,随着愈发严重的全球变暖,海水温度也受到了十分显著的影响。海洋生物对于海水温度十分敏感,每种生物的生长和繁殖都具有临界温度范围,温度通过激素和神经控制影响消化的酶反应。同时,温度的升高会对水中的有毒物质的毒性进行增强,驱使鱼群离开原本适宜的海域,前往具有更加适宜温度的海域。http://www.watercenter.org/physical-water-quality-parameters/water-temperature/water-temperature-effects-on-fish-and-aquatic-life/
The Gulf of Maine is one of the fastest reported ocean warming areas in the world, with water temperatures expected to rise again by 5.5 degrees Fahrenheit by 2065. Ocean warming has affected Atlantic salmon populations in the Penobscot, St. Croix, and Miramichi rivers in New Brunswick. Atlantic salmon is also migrating to deeper waters.\cite{REF2} 

% 缅因州湾是报道的世界海洋变暖最快的地区之一,预计到2065年,水温度将再次上升5.5华氏度。海洋变暖已经影响了Penobscot河,圣克罗伊河以及新不伦瑞克省Miramichi河的大西洋鲑鱼种群。大西洋三文鱼正往更深的水域迁移。(引用自https://www.chinaoceannews.cn/keji/128963.html)
Taking North Sea cod as an example, the average distribution of cod in the North Sea has also changed significantly over the past three decades, most of which have shifted to more northern and deeper waters. A strong influence on this phenomenon is the apparent warming trend in the North Sea. At the same time, this phenomenon has also led to a plunge in catches by the border countries in their original fishing grounds. However, the northward shift of fish stocks in the world has damaged some fishing grounds, but also spawned other fishing grounds.\cite{REF3} \cite{REFE} 
%       以北海鳕鱼为例,在过去的三十年中,鳕鱼在北海的平均分布也发生了明显的变化,大部分转移到了和更北端的和平均更深的水域。对于这种现象有一个很强的影响因素就是北海明显变暖的趋势。同时此现象也导致沿边国家在原有渔场的捕捞量暴跌。但全球鱼群北移破坏了部分渔场,但同时也催生了另外一些渔场。
% https://www.ncbi.nlm.nih.gov/pmc/articles/PMC4282283/

Global warming and the gradual increase in the temperature of the sea near Scotland have driven fish schools out of their original seas. This has a huge impact on fisheries production in Scotland. Therefore, we will analyze the changes in fish school positions over the next 50 years and what strategy we should take seems significant.
%        全球气候变暖,苏格兰附近海水温度逐渐升高,驱使鱼群离开原本适宜的海域,这对苏格兰的渔业生产造成了巨大影响,因此研究鱼群在未来50年内的位置变化以及对此应当采取的策略显得意义重大。

\subsection{Restatement of the Problem}

% 问题一:建立一个数学模型,以确定这两种鱼类在未来50年内最有可能出现的位置,假设水温变化足以导致种群移动。

% 问题二:根据海水温度变化,使用您的模型预测最佳情况和最坏情况,到时这些小渔业公司因距离鱼群太远将无法收获。

% 问题三:根据你的预测分析,这些小型渔业公司是否应该改变他们的经营?

% a、如果是,请使用您的模型来确定和评估对小型渔业公司具有实际和经济吸引力的战略。你的策略应该考虑但不限于现实的选择,包括:

%  -将渔业公司的部分或全部资产从苏格兰港口的当前位置转移到更靠近两个鱼类种群移动的位置;

% -使用一定比例的小型渔船,可以在没有陆地的情况下作业一段时间,同时仍然确保渔获物的新鲜度和高质量。

% b、如果您的团队拒绝任何更改的需要,请根据您的建模结果证明您拒绝的原因,因为它们与您的团队所做的假设有关。

% 问题四:如果某一比例的渔业进入另一个国家的领海(海洋),请使用您的模型来说明您的方案受到的影响。

% 问题五:除了你的技术报告外,为相关组织准备一到两页的文章,以帮助渔民了解问题的严重性,以及你提出的解决方案将如何改善他们未来的商业前景。
\begin{itemize}
  \item Build a mathematical model to identify the most likely locations for fishes over the next 50 years.
  \item Predict best case, worst case and most likely elapsed tiem(s) unitil small fishing companies unable to harvest.
  \item Whether these companiese should make changes to their operations. If yes, identify and assess strategies. If not, justify reasons.
  
 \item How our proposal is affected if some proportion of the fishery moves into the territorial waters(sea) of another country.
 \item Prepare an article for magazine to help fishermen.
\end{itemize}

\section{Problem Analysis}
Question 1: 

We need to build a mathematical model to determine where these two fish species are most likely to appear in the next 50 years. We need to first build a model to predict the changes in water temperature in the next 50 years. It is not necessary or possible to accurately predict the temperature of each point in the waters around Scotland. Therefore, we can divide some key large areas and combine our existing experience with previous data. The average annual temperature of these regions over the next 50 years is calculated. Water temperature change is the most significant factor to affect fish migration. In addition, fish migration is also related to the distance to the destination. The school of fish tends to migrate to an area where the water temperature is not much different from their favorite temperature. If the destination is near, the probability will also be larger. But it is difficult to directly discuss the change of fish school position in the next 50 years, computer simulation can be used to avoid the complicated internal mechanism. \cite{REF4}
% 问题一:我们需要建立一个数学模型,以确定这两种鱼类在未来50年内最有可能出现的位置。我们需要先建立模型预测未来50年水温会发生的变化,没有必要也不太可能精确预测苏格兰附近海域每一点的温度,因此可以划分一些关键的大区域,结合我们现有的经验和以往数据拟合出未来50年这些区域的年均温。水温变化足够影响鱼群的迁徙,另外鱼群的迁徙还与到目的地的距离有关,鱼群倾向与向水温与它们的最喜温度相差不大的海域迁徙,如果目的地较近,这个概率也会更大,但是如果直接讨论未来50年鱼群位置的变化显得有些困难,因此可以采用计算机模拟的方式,避开其中的复杂内部机理。

Question 2:

The migration of fish is due to changes in water temperature, some small fisheries companies will not be able to replenish their fish and the arrival of this situation is not simultaneous for different fishery companies, which is related to the direction of fish schools migration and the company's geographical location. We can divide the waters around Scotland into several large fisheris. According to the prediction results of the question 1, calculate the annual fish content of each fishery in the next 50 years. We believe that all small fishing companies in the fishing ground are in the same situation. If the fishery content has dropped to a certain degree, and we can regard that small fishing companies in this range will not be able to harvest. By counting the year of arrival at this time, determine the best case and worst case.
% 问题二:由于水温变化导致鱼群迁徙,一些小型渔业公司将会无法补到鱼,这一情况的到来对于不同的渔业公司并不是同时的,这与鱼群的迁移方向和公司的地理位置有关。我们可以把苏格兰附近海域划分为几个大的渔场,根据第一问的预测结果,统计出未来50年每个渔场每年的含鱼量,认为该渔场内的所有小型渔业公司处境相同,如果当年的渔场含鱼量下降到一定程度,我们可以认为该范围内的小型渔业公司将无法收获。通过统计这一时间的到来年份,确定最好情况和最坏情况。

Question 3: 

We need to identify and evaluate strategies that are practically and economically attractive to small fishing companies. The two options available include company relocation and the deployment of small fishing vessels. The advantages and disadvantages of these two strategies need to be considered comprehensively. As a typical risk decision-making problem, we need to first determine the goal of the decision, which is to maximize the company's profit in the future. The factors that need to be considered include the annual fishing volume, initial investment and subsequent investment. The annual fishing volume can be introduced based on the results of the previous questions. The initial investment in the company's relocation should be positively related to the distance of the relocation, and it is generally higher than the initial investment of small fishing vessels. Increased investment. Which strategy is adopted may also take into account the company's assets and other fishery companies' decisions.
% 问题三:我们需要确定和评估对小型渔业公司具有实际和经济吸引力的战略,可选的两种策略包括公司搬迁和配备小型渔船,这两种策略的优劣势需要综合考虑。作为典型的风险决策问题,我们需要首先确定决策的目标,即未来一段时间内公司利润最大化,需要考虑的因素包括每年的捕鱼量,初始投入和后续投入。其中每年的捕鱼量可以根据前几问的结果推出,公司搬迁的初始投入应该与搬迁的距离正相关,而且一般比配备小渔船的初始投入要高,但是小渔船数量的增加势必会造成后续投入的增加。具体采用哪种策略可能还要综合考量公司的资产情况,其它渔业公司的决策等。

Question 4:

If a certain proportion of fish schools enter the territorial sea of other countries, the fishery resources of this part will be lost, but the specific impact should be analyzed in combination with Scotland's geographical location and fish school distribution changes.
% 问题四:如若一定比例鱼群进入他国领海,则该部分渔业资源将会损失,但是具体影响要结合苏格兰的地理位置和鱼群分布变化进行分析。
\section{Assumptions and Symbols Definitions}
\subsection{Assumptions}

\begin{itemize}
\item Assumed that the number of herring or mackerel is generally unchanged, the fishing rate of humans is basically the same as the rate of fish reproduction.
\item Assumed that herring or mackerel's measures to cope with changes in water temperature are only horizontal position migration, and tend to migrate in the direction of the appropriate temperature.

\item Assumed that changes in water temperature and migration distance are sufficient to affect the probability of herring or mackerel populations migrating to a nearby location.

\item Assumed that the trend of global warming and ocean warming will not change in the next 200 years.

\item Assumed that small fishing companies have a limited operating radius and that all companies have the same operating radius.

\end{itemize}
% 1.假设鲱鱼或者鲭鱼的数量总体不变,人类的捕捞速度与鱼群繁衍速度基本持平。
% 2.假设鲱鱼或鲭鱼应对水温变化的措施只有发生水平位置迁徙,并倾向于向适宜温度的方向进行迁徙。
% 3.假设水温变化和迁徙路程足够影响鲱鱼或者鲭鱼种群迁徙到某一邻近位置的概率。
% 4.假设全球气候变暖海洋升温的趋势未来200年不发生改变。
% 5.假设小型渔业公司的作业半径有限,并且所有公司作业半径相同。

\subsection{Symbols Definitions}

\begin{center}
\begin{longtable}{p{.1\textwidth}p{.8\textwidth}m{.4\textwidth}}
\caption{The List of Notation}\\
\hline
Symbol& Meaning \\
\hline

$p(x,y)$      & Probability of fish school migrating to (x, y) position
                                                         \\
$t_0$      & Favorite temperature for herring or mackerel
                                                          \\
$t$     & Sea temperature
                                                        \\
$y$       & Year                                                           \\
$d$      & Distance of fish migration                                                            \\
$O_{ij}$       & The cost of company j in year i                                  \\
$I_{ij}$       & The income of company j in year i                                         \\
$q_{ij}$       & The number of units of fish in company j in year i
\\
$r$       & The monthly maintenance fee \\
$P_i$       & The munber of evacuees using bus to go from node i to node j at that time \\
$m$      & The distance of the company have moved                                                            \\
$n$       & The number of fishing vessels purchased                      \\
$g$       & The unit price of each small fishing vessel                     \\
$k$       & The company's migration price per unit distance
\\
$h$       & The income per unit of fish                                                                \\
$b$       & The efficiency improvement factor when using small fishing vessels                         \\
                                                        \\ \hline

 \end{longtable}
 \end{center}


\section{Establishment and Solution of Model}
\subsection{The Model for the First Question}
\subsubsection{The Regional Sea Temperature Model}
We divided the range \ang{52} N \textasciitilde \ang{62} N, \ang{0} W \textasciitilde \ang{10} W into 20 small squares into a $20 \times 10$ grid, as is shown in figure \ref{RealMap}.


\begin{figure}[H]
  \centering{
  \includegraphics[width=3cm]{./picture/1_1.png}}
  \caption{The Realistic Map Around Scotland}\label{RealMap}
\end{figure}

Then we abstracted the edges of the continent and the surrounding waters, as is shown in Figure \ref{AbstractMap}.
\begin{figure}[H]
  \centering{
  \includegraphics[width=3cm]{./picture/1_2.png}}
  \caption{The abstacted map of the continent and the surroundting waters around Scotland}\label{AbstractMap}
\end{figure}

The reader can compare the following figure \ref{FisheriesAround} with the information of the fishery we chose to discuss, which are shown in the table \ref{FisheryTable}.
\begin{figure}[H]
  \centering{
  \includegraphics[width=3cm]{./picture/1_3_1.png}}
  \caption{Fisheries around Scotland}\label{FisheriesAround}
\end{figure}

\begin{table}[H]
\centering
\begin{tabular}{|l|l|}% 通过添加 | 来表示是否需要绘制竖线
\hline  % 在表格最上方绘制横线
\textbf{Fishery Name}&\textbf{Representative Color}\\
\hline  %在第一行和第二行之间绘制横线
Argyll \& Clyde & Brown\\
\hline % 在表格最下方绘制横线
Orkney Islands & Yellow\\
\hline % 在表格最下方绘制横线
Outer Hebrides  & Purple\\
\hline % 在表格最下方绘制横线
Shetland Isles & Blue\\
\hline % 在表格最下方绘制横线
North Coast \& West Highlands & Green\\
\hline % 在表格最下方绘制横线
\end{tabular}
\caption{Represntive Color of Fisheries}
\label{FisheryColor}
\end{table}

Then we collected the global ocean temperature data from 2000 to 2017. We sorted out the ocean temperature distribution for each year in the range of \ang{52} N \textasciitilde \ang{62} N, \ang{0} W \textasciitilde \ang{10} W, and displayed it by a $20 \times 10$ matrix. The temperature data type of the ocean location is a float, and the data of the land part is set to NaN. The \ref{FP} is the processed matrix data for 2017.

\begin{figure}[H]
  \centering{
  \includegraphics[width=12cm]{./picture/FP.png}}
  \caption{The sea temperature matrix around Scotland in 2017}\label{FP}
\end{figure}

Then we analyze the characteristics of temperature changes from 2000 to 2017. First, we observed the annual average temperature changes from 2000 to 2017. We tried to process the image with polynomial fit and got the formula \ref{5}, and its graph is shown in figure \ref{FunctionGraph}
\begin{equation}\label{5}
  t = 0.001854 y^2 - 0.03 y + 10.3
  \end{equation}

  \begin{figure}[H]
    \centering{
    \includegraphics[width=10cm]{./picture/1_4.png}}
    \caption{Annual average temperature changes around Scotland from 2000 to 2017}\label{FunctionGraph}
  \end{figure}
Although the polynomial we got is not particularly good, we can observe the trend of higher water temperature in recent years from the figure. If we directly use this polynomial to predict the temperature data for the next 50 years, the data will be a bit out of expectation. Therefore, we did not use this formula to predict directly. Instead, we added the interval limit and noise processing.

According to information we collected, the temperature of the upper 2000 meters of the global ocean in 2019 is 0.075 degrees Celsius higher than the average state of 1981-2010. During the period 1987-2019, the average ocean warming rate was 450\% of that during the period 1955-1986, showing a continuous trend of accelerated ocean warming. 

Based on this information combined with existing data, we determined that the future temperature increase interval is 7 to 14 degrees Celsius. This interval is not strict. We will also consider the situation where the water temperature rises by 0.075 degrees Celsius every 20 years. The interval will be changed over time. For data that is directly fitted with polynomials, we will assign a certain weight ratio, and then add a random value with a normal distribution so that its final predicted value falls within the limit of the year. The final formula for predicting water temperature in the next 50 years is as follows:

\begin{equation}\label{2}
  t = a \cdot \ln (0.001854 y^2 - 0.03 y + 10.3) + s
  \end{equation}

Where $t$ is the predicted water temperature, $a$ is a constant, $y$ is the year, and $s$ is a random number with a normal distribution. The logarithmic function and the random number with a normal distribution are modifications to the polynomial fitting, so that the negative impact on the model was declined to the lowest. Then we use this model to predict the water temperature in the next 50 years. The water temperature maps for 2017 and 2070 are as follows.

As shown in Figure \ref{SeaT}, we can observe the depth of the color of the small grid to directly obtain the water temperature level. The dark blue area in the figure is the land area, and its value is meaningless.

\begin{figure}[H]
  \centering
  \subfigure[In 2017]{               %小图题的名称
  \includegraphics[width=4cm]{./picture/1_5.png}}
  \hspace{0in}
  \subfigure[In 2070]{
  \includegraphics[width=4cm,height=6.7cm]{./picture/1_6.png}}
  \label{SeaT}
  \caption{The Sea Temperature Map around Scotland in Different Years}
  \end{figure}
Using the model above, we could obtain a water temperature change map for the next 50 years. The map is composed of $20 \times 10$ squares. Next, we use computer simulation to let fish schools migrate according to water temperature changes and positional relationships. The 20 x10 matrix is obviously not enough to simulate the migration of this school of fish. For this purpose, a $200 \times 100$ matrix is first generated to represent the distribution of the school of fish. The matrix value is a Boolean number. True means fish, false means no fish, a NaN means it is on land and meaningless. We can see the fish production of various fisheris in Scotland in 2017 as shown in the table below.

\begin{table}[H]
  \centering
  \begin{tabular}{|l|l|}% 通过添加 | 来表示是否需要绘制竖线
  \hline  % 在表格最上方绘制横线
  \textbf{Fishery Name}&\textbf{Production Amount in 2017}\\
  \hline  %在第一行和第二行之间绘制横线
  Argyll \& Clyde & 37506\\
  \hline % 在表格最下方绘制横线
  Orkney Islands & 20956\\
  \hline % 在表格最下方绘制横线
  Outer Hebrides  & 30668\\
  \hline % 在表格最下方绘制横线
  Shetland Isles & 35947\\
  \hline % 在表格最下方绘制横线
  North Coast \& West Highlands & 30948\\
  \hline % 在表格最下方绘制横线
  \end{tabular}
  \caption{Fish production in each fishery around Scotland in 2017}
  \label{FisheryAmount}
  \end{table}

We initialized the fish school distribution according to the data, and generated some points in each fishery according to the proportion. The initial fish school distribution is shown in the figure \ref{InitDist}.

\begin{figure}[H]
  \centering{
  \includegraphics[width=5cm]{./picture/1_7.png}}
  \caption{The initial fish school distribution around Scotland}\label{InitDist}
\end{figure}

The density of fish in this area can be roughly judged by analyzing the density of the dot matrix in this erea.

In order to simulate fish school migration, we first determine that the probability is related to migration distance and water temperature change. We define the probability of fish school migration from one point to another as follows:
\begin{equation}\label{3}
      p(x,y) = \sigma (C_1 \cdot (t(x,y)-t_0)+C_2 \cdot d(x,y))
    \end{equation}
  \begin{equation}\label{4}
    \sigma (z) = \frac{1}{1+e^{-z}}
  \end{equation}
 
  where:
  
  $p(x,y)$ is the probability of fish school migration to (x, y) point ,
  
  $t(x,y)$ is the water temperature of the destination (x, y), 
  
  $t_0$ is the favorite temperature of herring or mackerel,
  
  $c_1,c_2$ is constants, 
  
  $d(x,y)$ is the distance from simulated point to (x, y) point, 
  
  A sigmod function is nested in the outer layer.
  
  This probability function can intuitively reflect the phenomenon that fish schools migrate to the sea with more suitable temperature as the water temperature changes, and the smaller the temperature difference, the more suitable the temperature at the destination. In addition, this probability is also related to the distance. If the migration distance is too far, the probability of fish migration will be lower. The probability function outer nested sigmod function is to limit the range to (0,1), to avoid the probability obtained is meaningless, which is inspired by the activation function in deep learning. The sigmod function curve is shown in figure \ref{sigmodcurve}.

  \begin{figure}[H]
    \centering{
    \includegraphics[width=5cm]{./picture/1_8.png}}
    \caption{The sigmode function curve}\label{sigmodcurve}
  \end{figure}
The next step is to use Matlab for computer simulation. The flow chart of the simulation is shown in the figure \ref{matlabflow}.

\begin{figure}[H]
  \centering{
  \includegraphics[width=8cm]{./picture/1_9.png}}
  \caption{The flow chart of the computer simulation}\label{matlabflow}
\end{figure}

The process of the simulation is the change of the distribution of fish schools per year, and we run the flowchart multiple times for one simulation to exclude the influence of accidental factors. We need to traverse each fish school, and then calculate the probability of the fish school migrating to a nearby point. After the probability of all surrounding points is calculated, a random number is generated. The final destination of this school of fish migration is determined according to which interval is the random number located.

\subsubsection{The Result of Modeling}
According to this simulation process and the water temperature change matrix obtained previously, we finally got a fish school distribution map for the next 50 years. For comparison, we display one every 10 years and add a real-world map as the background, as shown in figure \ref{Dist}.

\begin{figure}[H]
  \centering
  \subfigure[After 10 years]{               %小图题的名称
  \includegraphics[width=4cm]{./picture/1_10.png}}
  \hspace{0in}
  \subfigure[After 20 years]{
  \includegraphics[width=4cm]{./picture/1_11.png}}
  \hspace{0in}
  \subfigure[After 30 years]{
  \includegraphics[width=4.2cm]{./picture/1_12.png}}
  \hspace{0in}
  \subfigure[After 40 years]{
  \includegraphics[width=4cm]{./picture/1_13.png}}
  \hspace{0in}
  \subfigure[After 50 years]{
  \includegraphics[width=4cm]{./picture/1_14.png}}
  \label{Dist}
  \caption{The prediction of future fish school distribution around Scotland}
  \end{figure}




Notes:
\begin{itemize}
  \item The density of the small black dots in the picture reflects the density of the school of fish.
  \item We only study fish schools that may be harvested, and the fisheris far from the mainland are not shown.
  \item You may see some points falling on the land. This is because we have abstracted the outline of England's continent into a grid chart.
  \item From the picture, we can roughly observe that after every 20 years, the distribution of the school of fish will move one degree northward, about 110km.
\end{itemize}

\subsection{The Model for the Second Question}
\subsubsection{Preparations before modeling}
For fishery companies in the same location, in addition to the geographical location, the change in seawater temperature is an important influencing factor for fish production. Based on the simulation of different speeds of seawater temperature changes, we can roughly analyze the elapsed times when samll fishing companies can't havest any more.
\subsubsection{The Result of Modeling}
First, we calculated the number of fish in each area in the next 50 years based on the sea temperature change model and computer simulation, as shown in Figure\ref{FishProductionCurve}

\begin{figure}[tbp]
  \centering{
  \includegraphics[width=8cm]{./picture/2_1.png}}
  \caption{Fish production curve in each fishery around Scotland within 50 years}\label{FishProductionCurve}
\end{figure}

\begin{table}[!htbp]
  \centering
  \begin{tabular}{|r|c|c|c|c|c|}% 通过添加 | 来表示是否需要绘制竖线
  \hline  %在第一行和第二行之间绘制横线
   & \textbf{Argyll} & \textbf{Orkney} & \textbf{Outer} & \textbf{Shetland} &\textbf{North}\\
  \hline % 在表格最下方绘制横线
  2017 & 70 &35 &32&40&88\\
  \hline % 在表格最下方绘制横线
  2030 & 57&32&26&31&78\\
  \hline % 在表格最下方绘制横线
  2040  & 54&30&28&21&69\\
  \hline % 在表格最下方绘制横线
  2050 &28&19&32&24&56\\
  \hline % 在表格最下方绘制横线
  2060 &22&24&39&25&42\\
  \hline % 在表格最下方绘制横线
  2070 &8&19&35&31&31\\
  \hline % 在表格最下方绘制横线
  \end{tabular}
  \caption{Fish production matrix in each fishery around Scotland within 50 years}
  \label{FisheryTable}
  \end{table}

We found that the fish production of most fisheris is decreasing year by year, and it is related to the geographical location of the fisheris: the latitude of the fishery , the depth of the water area, whether there are other fisheris near the low latitudes and  whether the location is necessary for fish migration all affect the change of fish production in the fishery. We adjusted the parameters in the model to simulate the most likely situation, and the result is as the figure above.

At best, 50 years later, the Argyll \& Clyde fishing grounds in Scotland are the first to fail to harvest.

Assuming that the fish production of a certain fishery is lower than 20\% of the initial production, we regarde that the area will not be able to harvest. According to the statistical results, the production of southest fishery Argyll \& Clyde is less than 20\% of its initial value after about 50 years. From this we can see that if the Argyll \& Clyde fishery does not change its business strategy, its performance will continue to decline and eventually unable to operate. It can be observed from the figure that when the Argyll \& Clyde fishery is facing a crisis, the production of the Outer Hebrides fishery and the Shetland Isles fishery has increased slightly, indicating that some of the fish migrating to the north of the Argyll \& Clyde fishery have entered a certain area in the north, which is a better situation.

At worst, most fisheries, including Shetland Isles, will not be harvested after 200 years.

Based on the trend of global temperature warming and the data of the average sea temperature change in the sea area near Scotland in the past 5 years, we can conclude that the growth rate of its water temperature is increasing year by year. We have to consider the worst case, that is, almost all small fishing companies have no fish to be harvested. Based on this situation, we have adjusted the model parameters of seawater temperature changes to comply with the trend of accelerated global warming and extended the forecast time. In the previous model, we knew that every 20 years, the school of fish moved north by one latitude. Based on the adjusted model, we used the estimated method to get the worst case: 200 years later, most fisheries, including Shetland Isles, will not be harvested.

\subsection{Feasible Strategies for Local Fisheries Companies}

\subsubsection{Analysis and Formula}
We believe that the ultimate purpose of corporate decision-making is to maximize profitability in the next 50 years, and the migration of fish will lead to a decline in the profit of companies. Therefore, we think it is very important for a company to re-plan its strategy. We will give a risk decision-making model here to reflect the impact of different input ratios of multiple production factors on the final profit.

As for strategy one, with the migration of fish , fisheries companies' catches of fish may decrease year by year. It is a good strategy to move companies northward. The company's migration takes a lot of money, while increasing linearly with distance. But relocating the company once means that it will no longer be necessary to pay for it in the future.

The formula for the annual profit of this strategy is:

\begin{equation}\label{1}
  I_{ij} - O_{ij} = h * q_{ij}
  \end{equation}

The formula for the accumulative annual profit of this strategy is:

\begin{equation}\label{2}
  \sum_{i=y_0}^{y_f} (I_{ij} - O_{ij}) = \sum_{i=y_0}^{y_f} (h * q_{ij}) - k*m
  \end{equation}
  
As for strategy two, by purchasing small fishing vessels with independent fresh-keeping capacity, the company's operating efficiency can be improved, the fishing output and freshness can be increased, and the annual income can be increased. The cost of buying fishing vessels in the early stages of this decision is much less than the cost of relocating the company, but this also is an annual maintenance fee and wage increase.

The formula for the annual profit of this strategy is:

\begin{equation}\label{3}
  I_{ij} - O_{ij} = h * q_{ij} * (1+b*n) - n*r
  \end{equation}

The formula for the accumulative annual profit of this strategy is:

\begin{equation}\label{4}
  \sum_{i=y_0}^{y_f} (I_{ij} - O_{ij}) = \sum_{i=y_0}^{y_f} (h * q_{ij}* (1+b*n) - n * r ) - n * g
\end{equation}
  

Where:

$O_{ij}$ is the cost of company j in year i;

$I_{ij}$ is the income of company j in year i;

$h$ is the income per unit of fish;

$q_{ij}$=0 the number of units of fish in company j in year i;

$k$ is the company's relocating price per unit distance;

$m$ is the distance the company moved;

$b$ is the efficiency improvement module when using small fishing vessels;

$n$ is the number of fishing vessels purchased;

$r$ is the monthly maintenance fee;

$y_0$ is the beginning year;

$y_f$ is the ending year;

Except for the distance the company migrates and the number of fishing vessels purchased is determined by the company's decision, the other quantities are parameters that are not affected by human will.

\subsubsection{Proper Strategies to Take}
By rationally setting the parameters, we can judge the profitability of these two strategies in the next 50 years. It can be observed through the formula that if strategy one is adopted, although the initial revenue is not ideal, the cost of moving the company is a great burden. However, in the later period, due to the migration of fish, production gradually increased, and eventually the accumulated income was higher. With the second strategy, income was initially increased due to the use of new fishing vessels. However, when fish schools moved, not only output began to decline, but also the cost of additional small fishing vessels got higher.

Although the whole profit of first strategy is seemed to be higher than the second strategy in terms of final returns, the second strategy may be the optimal choice for companies with limited financing capabilities. The decisions of other companies may also have an impact on themselves. The aggregation of fishery companies will cause losses to each other. The flexibility of strategy two can come in handy. The specific strategy to be adopted depends on the actual situation of the company.

\subsection{Impact of foreign territorial seas on the benefits}
Observing Scotland's continental location, we can see that there are almost no territorial waters in other countries near the east-west direction. What we only need to consider is the territorial waters of the Irish territorial waters in the south and the Faroe Islands in the north. According to the previous research conclusions: about every 20 years, the school of fish moves northward by one degree of latitude, then about 50 years later, some fish will enter the waters of the Danish Faroe Islands. But the waters of the Faroe Islands are very far away from these fishing companies, so this part of the school of fish is not in our consideration. Therefore, if a certain proportion of fishery enters the territorial sea of another country, it will have little effect on the model we have established earlier.

\section{The Evaluation of Model and Further Discussion}

\subsection{Strengths}

\begin{itemize}
  \item From the perspective of animal ethology, it's reasonable to analyze the migratory habits of fish and establish modeling;
  \item Using the method of computer simulation and computer simulation through a large amount of data, easily avoiding the complicated internal mechanism of fish migration;
  \item In order to avoid accidental errors, the computer simulation was repeated 10 times each year;
  \item Establish a seawater temperature change model by collecting seawater temperature and known climate changes, which can perfectly predict the water temperature changes in the waters around Scotland;
  \item The parameters in the model can be adjusted according to the changes of the ecological environment, which is suitable for different sea areas;
\end{itemize}

\subsection{Weaknesses}
\begin{itemize}
  \item For the convenience of modeling, the water temperature distribution uses a 20X10 grid, which will inevitably affect the accuracy;
  \item It is difficult to avoid the influence of subjective consciousness because some parameters of the model are artificially set;
  \item Temperature changes are in annual units, without taking into account the effects of seasonal fish tour and seasonal fishing;
  \item Only the temperature change of seawater at 50m underwater was simulated, and the conditions of different water layers and the effects of ocean currents were not considered;
\end{itemize}

\subsection{Further Application and Extension}
According to our interpretation of the topic, we can know that our model is to predict the integral behavior of the group, and to simulate the integral behavior of the integral regularity by simulating the random behavior of the individual.

The situation described in this question is similar to the marine life in other areas in many cases. Therefore, this model has the potential of Scale-out first. It can not only predict the migration direction of Scottish herring and mackerel along the Scottish coast, but also The research object has been modified and promoted to predict the migration direction of marine animals in any sea area in the world, which can effectively provide fisheries companies in various places to the north of various natural fishing grounds, reduce the losses of fishery companies, and maintain the company as much as possible .

In addition, we can also expand our model vertically. Seawater temperatures are known to decrease with increasing depth, and deeper waters may be the second choice for mackerel and herring. But at the same time, the water pressure will gradually increase, so the habitat of mackerel and herring may change within a limited range and depth. If a depth consideration is added to the original model, the matrix used to describe the temperature of the sea and the distribution of the current fish will be expanded to three dimensions. Although the amount of calculation is increased by one dimension, the comprehensiveness and accuracy of the model's calculation results are increased. 
\newpage
\section{Article for Hook Line and Sinker Magezine}
\begin{center}
\textbf{What's Next for Scottland Fishing}
\end{center}

Scottish herring and mackerel contribute greatly to the economy of Scottish fisheries. In order to better understand the related issues that these two fish may migrate from existing habitats near Scotland, our team have done some research and hope to help Scottish fishermen understand how seriousness of  the problem. In addition, we have proposed some solutions, hoping to improve their future business prospects.

Based on computer simulation methods, our team predicted the most likely locations for these two fish species in the next 50 years based on water temperature changes over 50 years and the habits of herring and mackerel. In addition, by simulating the profit and loss data of small fisheries companies choosing different strategies for a period of time in the future, we identify and evaluate strategies that are practical and economically attractive to small fisheries companies.

The result of our technical analysis is: With global warming, schools of fish in Scottish waters are moving north by 1 degree latitude every 20 years.

Our team have compiled the water temperature data for Scotland from 2000 to 2017, and then analyzed the data to find that the global water temperature has a rising trend, and this trend has become faster and faster in recent years. The temperature of the upper 2000 meters of the global ocean in 2019 is 0.075 degrees Celsius higher than the average state of 1981-2010. The heat content of the upper 2000 meters of the ocean in 2019 is 25 * 1021 joules higher than in 2018, setting a new historical record. During the period 1987-2019, the average ocean warming rate was 450\% during the period 1955-1986, showing a continuous trend of accelerated ocean warming. Marine organisms are very sensitive to the temperature of seawater. Each organism has a critical temperature range for its growth and reproduction. The temperature controls the enzyme reactions that affect digestion through hormones and nerves. At the same time, the increase in temperature will increase the toxicity of toxic substances in the water, driving fish away from the originally suitable sea area to the sea area with a more suitable temperature. In the southern part of Scotland, there will be no fish to supplement due to the decrease in the number of fish. About 50 years later, Argyll \& Clyde fishing grounds will not be harvested. About 200 years later, most fishing grounds including Shetland Isles fishing ground It will not be harvested, and Scottish fisheries production will cause huge losses.

As for this case, our team have proposed two feasible solutions  in order to improve their future business prospects.

The first solution is to transfer some or all of a fishing company's assets from a current location in a  Scottish port to a closer to where both fish populations are moving; the second solution is to use a certain percentage of small fishing vessels to expand the fishing company's operating radius to ensure the fish's freshness and quality. 

We identify and evaluate strategies adopted by small-scale fisheries companies based on  risk decision-making model. We found that the goal of the decision was to maximize profits in the next 50 years, and then combined the changes in the distribution of the fish school to calculate the profit for each of the two solutions. We found that the initial investment in the relocation of the solution one company is relatively large, but the subsequent increase in output can make up for this loss. In the long run, solution one is better than solution two. The initial investment of solution two is related to the number of small fishing vessels. This investment is less than the initial investment of solution one. It is a feasible solution for enterprises with insufficient assets. As for solution two, due to the increase in fishing equipment, the company's annual  expenditure will be more High, and fish catches will still slowly decrease as the fish school moves north. 

In addition, solution one may be better in the long run, but companies with small volumes can consider solution two. The decisions of other companies may also have an impact. The agglomeration of companies may cause losses to each other and the flexibility of solution two can come in handy. It is useful to decide which strategy to adopt based on the actual situation of the company.

The water temperature in Scotland is warming, and the school of fish is slowly migrating to the north. It is hoped that fishing companies which are aware of the seriousness of the problem can choose appropriate strategies based on actual conditions to improve the future fishing environment in Scotland.

\newpage
\addcontentsline{toc}{section}{Reference}
\bibliographystyle{plain}
\bibliography{myreference}



\end{document}
